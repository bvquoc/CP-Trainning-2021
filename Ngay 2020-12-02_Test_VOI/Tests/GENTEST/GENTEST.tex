\begin{problem}{GENTEST}{GENTEST.inp}{GENTEST.out}{1 giây}{512M}
\small{
\addcontentsline{toc}{subsection}{GENTEST}

Ban ra đề đang viết chương trình sinh test cho một bài về đồ thị. Họ cần sinh một đơn đồ thị vô hướng $n$ đỉnh $m$ cạnh. Thuật toán sinh như sau:
\begin{enumerate}
	\item Nếu đã lấy đủ $m$ cạnh thì dừng.
	\item Gọi $E$ là tập các cặp $(u,v)$ với $u<v$ và $u$ đang không kề với $v$, tức $E$ là tập các cạnh chưa có trong đồ thị.
	\item Sắp xếp $E$ theo thứ tự từ điển ($(u,v)$ đứng trước $(x, y)$ nếu hoặc là $u<x$ hoặc là $u=x$ và $v<y$).
	\item Chọn một số tự nhiên $i$ trong phạm vi từ $1$ đến $|E|$.
	\item Nạp cạnh thứ $i$ trong $E$ vào đồ thị, lặp lại bước 1.
\end{enumerate}
Cho biết các số được chọn ở bước thứ 4, hãy giúp ban ra đề in ra các cạnh của đồ thị. Lưu ý, các loại chỉ số trong bài đều được đánh số bắt đầu từ $1$.

\InputFile
\begin{itemize}
	\item Dòng đầu chứa hai số nguyên dương là số đỉnh và số cạnh của đồ thị: $n$ $m$
	\item $m$ dòng tiếp theo, dòng thứ $t$ chứa một số nguyên dương là số được chọn ở bước thứ 4 của lần lặp thứ $t$: $i_t$
\end{itemize}

\OutputFile
In ra $m$ cạnh được chọn theo thứ tự chọn của thuật toán. Với mỗi cạnh, in ra hai đỉnh trên một dòng cách nhau bởi dấu cách, đỉnh bé hơn phải được in trước.

\Examples
\begin{example}%
\exmp{
4 4
3 3 1 3
}{
1 4
2 3
1 2
3 4
}%
\end{example}
\Explanation
\begin{itemize}
	\item Ban đầu $E = \{(1,2), (1,3), (1,4), (2, 3), (2, 4), (3, 4)\}$ nên cạnh thứ $3$ là $(1, 4)$
	\item Tiếp theo $E = \{(1,2), (1,3), (2, 3), (2, 4), (3, 4)\}$ nên cạnh thứ $3$ là $(2, 3)$
	\item Tiếp theo $E = \{(1,2), (1,3), (2, 4), (3, 4)\}$ nên cạnh thứ $1$ là $(1, 2)$
	\item Tiếp theo $E = \{(1,3), (2, 4), (3, 4)\}$ nên cạnh thứ $3$ là $(3, 4)$
\end{itemize}

\Scoring
\begin{itemize}
	\item Trong tất cả các test: $1 \leq n \leq 10^9$, $1 \leq m \leq 10^5$
	\item Có 10\% số test với: $1 \leq n, m \leq 500$
	\item 30\% test tiếp theo với: $1 \leq m \leq 500$
	\item 30\% test tiếp theo với: $1 \leq n \leq 500$
	\item 30\% test tiếp theo với ràng buộc gốc
\end{itemize}

\begin{comment}
\Hint

\end{comment}
}
\end{problem}
